% !TeX spellcheck=en_US

\section{Experiment Result}\label{Experiment Result}
In this section, we conduct extensive experiments on real-world trajectory datasets to study the performance of the proposed algorithms, by comparing the features of the generated trajectory dataset with the original dataset $\mathcal{D}$. All algorithms are implemented on a Core i5-3470 3.20GHz machine with 16GB memory and a windows platform.

\subsection{Dataset}

In order to comprehensively display the performance of the proposed model FTS, we use a real-world trajectory dataset in Beijing, which contains 13245 taxi trajectories (1401413 GPS points) that collected from 1752 taxis during 1/10/2013-7/10/2013. Note that we use a $200\times400$ grid map on Beijing to generate 14500 new trajectories.


\subsection{Performance Evaluation}
\subsubsection{Solutions}
We evaluate the proposed algorithms by comparing the trajectory features of the given dataset $\mathcal{D}$ with that of synthesized trajectories. Following methods are used:   1) Baseline 1: random generation (RG), 2) Baseline 2: intersection partition (IP), 3) Approach 3: gene partition (GP). Moreover, we use Original to denote the given trajectory dataset $\mathcal{D}$.

% \textbf{Distribution of length}. First of all, we compare the distribution of trajectory length between original dataset and generated datasets. Seen from Fig.\ref{dis of length bj}, RG performs worst and it is more likely to generate long trajectories, as this approach has no restriction during synthesis. IP performs better than RG, since we have proposed several functions to control each synthesis. Note that, GP performs best, since it has the most similar distribution to the original dataset. This is because GP decomposes trajectories reasonably, then it is more likely to synthesize similar trajectories.
% \begin{figure}[H]
% 	\centering
% 	\includegraphics[width=1.0\textwidth]{figs/length_bj.eps}\\
% 	\caption{Distribution of trajectory length in Beijing}
% 	\label{dis of length bj}
% \end{figure}
% \begin{figure}[H]
% 	\centering
% 	\includegraphics[width=1.0\textwidth]{figs/speed_bj.eps}\\
% 	\caption{Distribution of trajectory speed in Beijing}
% 	\label{dis of speed bj}
% \end{figure}

% \textbf{Distribution of speed}. Speed is of critical importance to reflect the traffic conditions of a city and describe motion behaviors of moving objects. Shown in Fig.\ref{dis of speed bj}, most of historical trajectories have speed 3m/s-8m/s in Beijing. The distribution of speed of trajectories generated by RG is different from the given trajectories, and IP performs better than RG. The trajectory speed of GP concentrate on 4m/s-8m/s, which is similar to the original dataset.

% \textbf{Distribution of U-turn}. Trajectory U-turn is another important feature that directly indicates the traffic quality and how the moving objects travel. Seen from Fig.\ref{dis of uturn bj}, GP has created the most similar trajectories to the given dataset, where most of synthesized trajectories have a small number of U-turn.

% \textbf{Distribution of acceleration}. We also investigate the performance of the propose approaches by comparing distribution of trajectory acceleration of corresponding datasets. Without surprise, in Fig.\ref{dis of accelration bj}, IP performs better than RG and GP performs best. This is because we have designed several heuristic functions to select reasonable sub-trajectories to make a synthesis, and merge sub-trajectories with similar features. Consequently, we are more likely to synthesize new trajectories with similar features in IP and GP.

% \begin{figure}[H]
% 	\centering
% 	\includegraphics[width=1.0\textwidth]{figs/uturn_bj.eps}\\
% 	\caption{Distribution of U-turn in Beijing}
% 	\label{dis of uturn bj}
% \end{figure}
% \begin{figure}[H]
% 	\centering
% 	\includegraphics[width=1.0\textwidth]{figs/acceleration_bj.eps}\\
% 	\caption{Distribution of trajectory acceleration in Beijing}
% 	\label{dis of accelration bj}
% \end{figure}



% \begin{figure*}[H]
% 	\centering
% 	\includegraphics[width=1.0\textwidth]{figs/density_bj.eps}\\
% 	\caption{Distribution of density in Beijing}
% 	\label{dis of density bj}
% \end{figure*}

% \textbf{Distribution of density}. In oder to display the generated trajectories intuitively, we plot all corresponding locations in Fig.\ref{dis of density bj}. Note that, different from aforementioned features, by which we can distinguish the performance of different methods easily, all methods perform well in this feature. This is because the large number of POIs in synthesized trajectories that have covered most area of a city. As with the original dataset, the synthesized trajectories concentrate on the center of Beijing.

% \subsubsection{Supplement}
% As shown in the Section 7,we think about the problem of trajectory generation on the contrary: 1) Merging the historical trajectories firstly. 2) Then cutting the merged long trajectory in a specific condition. Thus, we perfect the problem of trajectory generation in two opposite ways. The distribution of features can be seen in the figure:
% The methods in Section 7 have the similar features with original data except the feature of speed shown in figure. Obviously, this method has certain limitation. The features of speed may be lost in generation.
% \begin{figure}[H]
% 	\centering
% 	\includegraphics[width=1.0\textwidth]{figs/supple.jpg}\\
% 	\caption{Distribution of trajectory features(supplement)}
% 	\label{supplement}
% \end{figure}
% \begin{figure}[H]
% 	\centering
% 	\includegraphics[width=1.0\textwidth]{figs/origin.jpg}\\
% 	\caption{Distribution of trajectory features(origin)}
% 	\label{origin}
% \end{figure}


\subsubsection{Efficiency} We also evaluate the efficiency of the proposed methods by comparing the running time of them. Without surprise, the method RG needs least running time, as it just synthesizes crossing trajectories directly and the number of candidates is less than that of IP and GP. Furthermore, Table \ref{running time} shows GP is much more efficient than IP, since the number of sub-trajectories generated by GP is much less than that generated by IP shown in table \ref{number of sub-tra}. The supplement method needs less time. However, the features of speed of origin data may lose in generation.
%, which shows the necessity of the optimization in Section \ref{Optimization}.
\begin{table}[H]
	\centering
	\caption{Running time}
	\label{running time}
	\begin{small}
		\begin{tabular}{p{3cm}|p{1.5cm}|p{1.5cm}|p{1.5cm}|p{1.7cm}}
			\hline
			Method & RG & IP & GP &Supplement \\ \hline
			Time cost (h:mm:ss) & 0:04:02 & 3:01:24 & 1:21:32 & 0:31:19 \\\hline
		\end{tabular}
	\end{small}
\end{table}
