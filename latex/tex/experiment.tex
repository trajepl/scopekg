% !TeX spellcheck=en_US

\section{Experiment Result}\label{Experiment Result}
In this section, we conduct extensive experiments on dataset which is generated 
randomly based on the information of some soccers player and their careers to 
evaluate the proposed methods for storing, indexing and query processing in scope 
query we porposed.

\subsection{Dataset}
We use the basic dataset in which per line describes that a soccker player serve a team
at a short time interval. Besides, the basic dataset also contains some  extra information 
such as the birthday of player and the his position in a soccer team and so on. The 
structure of the dataset is described in Table \ref{basic_data}.

\begin{table}
	\centering
	\caption{Soccer player information example}
	\label{basic_data}
	\begin{small}
		\begin{tabular}{p{1.6cm}|p{2.5cm}|p{1.2cm}|p{1.3cm}|p{1.7cm}|p{1.5cm}|p{1.5cm}}
			\hline
			Player & Team & Position & Birthday & Birth Place & Start Year & End Year\\ \hline
			Ant & Houston Texans & DB & 1991 & Florida & 2016 & 2020 \\\hline
			A.J. Edds & New York Jets & LB & 1987 & Iowa & 2014 & 2018 \\\hline
			Korey Hall & Chicago Bears & DB & 1983 & Boise State & 2014 & 2015 \\\hline
		\end{tabular}
	\end{small}
\end{table}

In order to comprehensively display the performance of the proposed n methods for different query, 
we extend the basic dataset generating new datasets which contains more entries about different 
soccer players. Different size of new generated datasets are described in Table \ref{gen_data}. Note that 
the number of players in generated dataset is different in multiple experiments because of the randomness 
of generation algorithms.

\begin{table}
	\centering
	\caption{Generated Dataset Description}
	\label{gen_data}
	\begin{small}
		\begin{tabular}{p{4cm}|p{3cm}}
			\hline
			dataset number of entry  & number of player \\\hline
			10,000 & 25,497 \\\hline
			100,000 &  253,784 \\\hline
			10,000,000 & 2,540,461 \\\hline
		\end{tabular}
	\end{small}
\end{table}


We implement storing, indexing and four kinds of basic queries processing as well as generating new dataset in Python.
To make comparison, we use non-index query processing as a baseline, i.e., for the several basic kinds of queries, the 
storage model is directly accessed and sequential scans are acrried out. And Then we use our hash methods in  memory 
and build index for time interval to process queries. The experiments are conducted on a PC with Intel Core i7-7700K 
CPU 4.20GHz Core 8, 16GB memory and 910GB disk on Linux version 4.14.3-1-ARCH.

\subsection{Baseline}
In this part, we sum the time of four kinds of queries which processing 10 queries respectively. As show in Fig.
\ref{baseline}, the cost in queries have a sharp increase sadly following the size of dataset increasing. The 
ticks in X axis denote the logarithms of low 10. It is obviously that the queries suffer poor performance in 
baseline.

\begin{figure}[H]
	\centering
	\includegraphics[width=1.0\textwidth]{pic/baseline.pdf}\\
	\caption{Baseline: Four kinds of queries time}
	\label{baseline}
\end{figure}

\subsection{Hash-based}
After building hash index for players and teams, and loading the relation existing in time interval into memory, 
we evaluate the same benchmark as well as in method of baseline. The trend in Fig.\ref{time} looks similar among four 
kinds of queries, but you should note that the triks in y axis have droped down less than one second from dozens of 
seconds in query HT, HR, HRT. But for the query REL processing, the cost is stil so high because of if we want to get 
what has happed in a time interval, we still need going through all the entries between players and teams.

\begin{figure}[H]
	\centering
	\includegraphics[width=1.0\textwidth]{pic/time_idx.pdf}\\
	\caption{Hash-based: Four kinds of queries time}
	\label{time}
\end{figure}

\subsection{Index for Time Interval}
We just build sequential index easily to solve the poor performance of REL query in the before method. The performance 
of REL query gets a lot improve which reduce time cost from 100-plus seconds to less than one second as show in Fig.
\ref{time-idx}. For showing the  extensibility of our methods, we add the four group test queries which have 10, 100, 
1000, 10000 entries respectively. In the extensibility experiment, we use the logarithms in the horizontal axis following
the nolinear increasement in number of queries. Therefore even if the trend in Fig.\ref{query} looks like exponential type, 
it is stll linear time increasement. Besides the cost in queries is still so slow that we can get the results of queries less
than one second even processing 10000 queries.
\begin{figure}[H]
	\centering
	\includegraphics[width=1.0\textwidth]{pic/time.pdf}\\
	\caption{Index-time-interval: Four kinds of queries time}
	\label{time-idx}
\end{figure}

\begin{figure}[H]
	\centering
	\includegraphics[width=1.0\textwidth]{pic/query.pdf}\\
	\caption{Four kinds of queries time}
	\label{query}
\end{figure}
Moreover we summary the running time of four basic query in different size datasets using line chart. Due to the high time cost 
in baseline, we just sum the 10 queries running time. As show in Fig.\ref{combine}, it is obviously the baseline need far more time
than the other methods. The method of hash-based performs wonderful except in query REL. As described before,  we still need going 
through all the entries between players and teams to get the cluster of event in a given time interval. After we build sequential
index for time interval, this predicament in query REL has been solved.





\begin{figure}[H]
	\centering
	\includegraphics[width=1.0\textwidth]{pic/combine.pdf}\\
	\caption{Four kinds of queries time}
	\label{combine}
\end{figure}

\begin{figure}[H]
	\centering
	\includegraphics[width=1.0\textwidth]{pic/index.pdf}\\
	\caption{Time of building index}
	\label{index}
\end{figure}

% \textbf{Distribution of speed}. Speed is of critical importance to reflect the traffic conditions of a city and describe motion behaviors of moving objects. Shown in Fig.\ref{dis of speed bj}, most of historical trajectories have speed 3m/s-8m/s in Beijing. The distribution of speed of trajectories generated by RG is different from the given trajectories, and IP performs better than RG. The trajectory speed of GP concentrate on 4m/s-8m/s, which is similar to the original dataset.

% \textbf{Distribution of U-turn}. Trajectory U-turn is another important feature that directly indicates the traffic quality and how the moving objects travel. Seen from Fig.\ref{dis of uturn bj}, GP has created the most similar trajectories to the given dataset, where most of synthesized trajectories have a small number of U-turn.

% \textbf{Distribution of acceleration}. We also investigate the performance of the propose approaches by comparing distribution of trajectory acceleration of corresponding datasets. Without surprise, in Fig.\ref{dis of accelration bj}, IP performs better than RG and GP performs best. This is because we have designed several heuristic functions to select reasonable sub-trajectories to make a synthesis, and merge sub-trajectories with similar features. Consequently, we are more likely to synthesize new trajectories with similar features in IP and GP.

% \begin{figure}[H]
% 	\centering
% 	\includegraphics[width=1.0\textwidth]{figs/uturn_bj.eps}\\
% 	\caption{Distribution of U-turn in Beijing}
% 	\label{dis of uturn bj}
% \end{figure}
% \begin{figure}[H]
% 	\centering
% 	\includegraphics[width=1.0\textwidth]{figs/acceleration_bj.eps}\\
% 	\caption{Distribution of trajectory acceleration in Beijing}
% 	\label{dis of accelration bj}
% \end{figure}



% \begin{figure*}[H]
% 	\centering
% 	\includegraphics[width=1.0\textwidth]{figs/density_bj.eps}\\
% 	\caption{Distribution of density in Beijing}
% 	\label{dis of density bj}
% \end{figure*}

% \textbf{Distribution of density}. In oder to display the generated trajectories intuitively, we plot all corresponding locations in Fig.\ref{dis of density bj}. Note that, different from aforementioned features, by which we can distinguish the performance of different methods easily, all methods perform well in this feature. This is because the large number of POIs in synthesized trajectories that have covered most area of a city. As with the original dataset, the synthesized trajectories concentrate on the center of Beijing.

% \subsubsection{Supplement}
% As shown in the Section 7,we think about the problem of trajectory generation on the contrary: 1) Merging the historical trajectories firstly. 2) Then cutting the merged long trajectory in a specific condition. Thus, we perfect the problem of trajectory generation in two opposite ways. The distribution of features can be seen in the figure:
% The methods in Section 7 have the similar features with original data except the feature of speed shown in figure. Obviously, this method has certain limitation. The features of speed may be lost in generation.
% \begin{figure}[H]
% 	\centering
% 	\includegraphics[width=1.0\textwidth]{figs/supple.jpg}\\
% 	\caption{Distribution of trajectory features(supplement)}
% 	\label{supplement}
% \end{figure}
% \begin{figure}[H]
% 	\centering
% 	\includegraphics[width=1.0\textwidth]{figs/origin.jpg}\\
% 	\caption{Distribution of trajectory features(origin)}
% 	\label{origin}
% \end{figure}
